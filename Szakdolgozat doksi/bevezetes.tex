	
\chapter{Bevezetés}	
%\addcontentsline{toc}{chapter}{Bevezetés} 
 
	
	
		
				
		Szakdolgozatom alapjául a radiális bázisfüggvényekkel való közelítés részletes, és érthető bemutatását választottam. Ehhez célom könnyen kezelhető program létrehozása, mellyel bármely felhasználó könnyedén kipróbálhatja saját adatokkal a függvényközelítést. Igyekeztem törekedni olyan jelöléseket, és értelmezéseket használni, melyet legtöbb ember egyszerűen megérthet. A programom funkciói elsősorban a közelítőmódszerekkel meghatározott függvények bemutatására szolgálnak, nem kimondottan konkrét helyek hőtérkép szerinti meghatározására.
		
		\begin{figure}[ht]
			\centering
			\includegraphics[scale=0.8]{main/program.png}
		\end{figure} 
		
		\section*{Köszönetnyilvánítás}
		Az interpolációs témaválasztásommal kapcsolatban szeretnék köszönetet nyilvánítani Dr. Lócsi Leventének, a rengeteg érdekes téma felkínálásáért, ezenfelül a nagymértékű segítségért, amelyet hasznosítani tudtam a program megírása során. Szeretném továbbá megköszönni, hogy kitartott mellettem témavezetőként, és hogy felhívta figyelmemet a dolgozatban szereplő hibákra.
		
	
		
	
		
		